% ============================
\section{Introduction}
% ============================

The video game industry is highly competitive, and commercial outcomes vary widely across titles. This project asks: \emph{Which factors best explain the commercial success of a game?} The goal is both descriptive (identify key drivers) and predictive (estimate sales performance). We focus on modern games (2005+) to reduce structural shifts in the market and to reflect current platform dynamics.

Beyond purely predictive accuracy, the project emphasizes interpretability. Sales can be driven by a mix of perceived quality, attention, and distribution channels. By combining interpretable models and feature importance with clustering-based exploration, the report aims to provide a concise explanation of success drivers that is useful for decision-making (publishers, platforms, or investors).

% ============================
\subsection{Literature Review}
% ============================

\subsubsection{Relevant preliminary work and theoretical background}
\paragraph{Research question and motivation}
Which factors explain the commercial success of a video game? This question matters because development and marketing costs are high, outcomes are uncertain, and platforms/publishers must allocate resources under strong risk. Understanding the drivers of success supports better greenlight, marketing, and release decisions.
The video game market is also characterized by strong ``winner-takes-most'' dynamics, where a small number of titles capture a large share of sales. This structural uncertainty makes success prediction especially challenging and relevant.

\paragraph{Evidence from the literature}
Greenwood-Ericksen (2013) shows that Metacritic scores are significantly correlated with video game sales, with critic scores generally more stable and informative than user scores \cite{GreenwoodEricksen2013}. The study frames Metacritic as a useful aggregate signal of reception, while acknowledging that it does not capture all determinants of sales (e.g., marketing, brand effects).
This motivates treating review variables as informative signals rather than causal drivers, and it aligns with a predictive framing of the problem.

\paragraph{Why success prediction is difficult}
Commercial success is shaped by uncertainty and non-linear dynamics. The market is characterized by heavy tails and ``winner-takes-most'' outcomes, where a small number of titles capture a large share of revenue. This makes prediction difficult because average patterns can be dominated by a few outliers.

\paragraph{Signals, attention, and timing}
Success is influenced not only by perceived quality but also by attention and timing. Release windows, platform reach, and publisher visibility can amplify or dampen the effect of reviews. This motivates including both reception indicators and categorical market descriptors in the feature set.

\paragraph{Link to the project design}
The literature suggests that review signals matter, but that they are entangled with popularity. This project therefore combines predictive modeling with careful discussion of endogeneity, and uses interpretable models to highlight which signals are most associated with sales.

\paragraph{Methodological caution}
Review-based variables are strong predictors of sales, but they must be interpreted with caution due to potential endogeneity and post-release effects. Popular games attract more attention and reviews, so the relationship is likely correlational rather than strictly causal.

\paragraph{Connection to this project}
This literature motivates the inclusion of critic and user scores, as well as their corresponding review counts. We treat these variables as predictive signals rather than causal drivers and emphasize the distinction between prediction and explanation.

\paragraph{Positioning}
Relative to prior work, this project compares multiple ML models, provides interpretable feature importance, and reports a cautious explanatory reading that acknowledges potential endogeneity in review signals.
