\section{Conclusion and Future Work}
\label{sec:conclusion}

\subsection{Summary}

This project analyzed and predicted video game success using a combination of regression, classification, and clustering. The best predictive model for sales was HistGradientBoosting, while Random Forest feature importance provided interpretable insights into what drives commercial performance.

The main finding is that visibility and engagement (critic/user counts) are the strongest predictors of sales, followed by platform and publisher effects. Review scores matter, but their impact is smaller than the volume of attention.

Limitations include reliance on historical sales data and the absence of marketing spend or advertising variables, which likely influence success. Future work could incorporate richer marketing signals, franchise indicators, or regional sales breakdowns, and explore model calibration for better success classification.

Despite these limitations, the project provides a reproducible pipeline and a compact set of interpretable figures that answer the research question. The combination of predictive accuracy and explanatory insights makes the analysis suitable for academic evaluation and practical discussion.

\subsection{Future Directions}
Future work could incorporate pre-release signals (e.g., wishlist activity or trailer engagement) and causal methods to mitigate endogeneity. Additional extensions could add marketing proxies or franchise indicators to improve external validity and interpretability. Exploring threshold choice and calibration could also improve recall for rare hit titles.
Another useful extension would be a sensitivity analysis on the 1M success threshold to test how conclusions shift across alternative business definitions of “success.”
Finally, comparing results across time windows (e.g., 2005--2010 vs.\ 2011--2016) would reveal whether predictive relationships are stable or shift with platform cycles.
