% ============================
\section{Appendix}
% ============================

\subsection{AI Usage}
This project used AI tools (ChatGPT) as a learning aid for refactoring,
debugging, and improving report wording/clarity. AI assistance was used to
propose code structure, suggest evaluation visuals, and rewrite paragraphs
for concision and academic tone. All final decisions, experiments,
interpretations, and submissions were made by the author, who verified the
outputs and ensured compliance with the course requirements.

\subsection{Cluster Profiles (Condensed)}
Cluster~1: High-engagement, high-sales titles with stronger critic reception.\\
Cluster~0: Lower-sales games with modest critic scores and mixed platforms.\\
Cluster~2: Mid-range group with slightly higher sales than Cluster~0 and older average release year.

\subsection{Code Repository}
\noindent
\textbf{GitHub Repository:} \url{https://github.com/maximesudan-hub/Video_Games_Success1}

\paragraph{Repository structure}
The repository follows the course template: \texttt{main.py} is the entry point; core logic lives in \texttt{src/} (data loading, models, evaluation); raw and processed data are under \texttt{data/}; outputs (figures and metrics) are written to \texttt{results/}.

\paragraph{Installation}
Create a virtual environment and install dependencies: \texttt{pip install -r requirements.txt}. The project uses Python 3.10+ and standard data-science libraries (pandas, scikit-learn, matplotlib).

\paragraph{Reproducibility}
Run \texttt{python main.py} from the repository root to reproduce all metrics and figures. The preprocessing can be rerun via \texttt{python -m src.data\_loader}. Random seeds are fixed (e.g., \texttt{random\_state=42}) to ensure consistent results.
